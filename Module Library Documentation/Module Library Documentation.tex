\documentclass[11pt]{book}

\title{Audulus 3 Module Library Documentation}
\author{Mark Boyd}



\setlength{\parindent}{2em}
\setlength{\parskip}{0em}

\begin{document}

\maketitle

\tableofcontents
\chapter{Modules and How They Work}

Modular synthesis is an exciting, freeing way to create music and design sounds. The freedom it offers can be intimidating at first, especially to a total beginner, but grasping the fundamentals is easier than it appears.

All Audulus modules are created using Audulus nodes. You can open up any module to see how it works. Some modules are very simple and may have only a few nodes inside. Others are a web of interconnected submodules, sometimes many layers deep.

The great thing about the Audulus module library is that you don't need to understand how each module is constructed to be able to use it. There are, however, a few things you need to know about the module library before getting started.

\section{Signal Standards}

Modules in the Audulus library operate with a set of standardized signals. The job of the modules is to generate and modify these signals in interesting ways.

There are four main signal types: Gate, Modulation, 1/Octave, and Audio. As a general rule, you will connect like with like, meaning connect a Gate output to a Gate input, or an Audio output to an Audio input.

\subsection{Gate}

The Gate signal is used to drive sequencers, open and close envelopes, and syncronize tempo, among other things. Its signal range is 0 or 1: off or on.

Gates are generated mainly by clock modules and keyboard or MIDI input. You can modify gate signals with modules like clock dividers and multipliers, gate sequencers, and Bernoulli gates. They are used by modules like sequencers, envelopes, and LFOs.

Gate inputs and outputs are always marked with a green light which is also called the Light node. When a gate input or output is equal to 0, the color of the light is dark green, and when the output is equal to 1, the color of the light is a bright green.

The only exception to this rule is when a gate is entering an envelope. Here, gate height, or the upper value of the gate, will set the maximum value that the Attack period rises to. This allows you to play with dynamics, or making your sounds softer or louder.

Gate inputs and outputs may be marked Gate, Gte, Reset or Rset, PWM, or Clk. Sometimes they may be unmarked except for the light node inside each input or output. These labels will have contextual meanings for each specific module, which are explained in their individual entries in the manual below.

Most modules will respond to only the rising edge of a gate. The rising edge is the moment where a gate transitions from 0 to 1. For example: a step sequencer will only step forward at the rising edge, but will not step forward again on the falling edge.

Other modules, like an envelope, respond to both the rising and falling edge of the gate signal. The rising edge of the gate will initiate the attack stage of the envelope whereas the falling edge will initiate the release stage.

If you are familiar with modular synthesis, you might be wondering if there is a difference between a pulse and a gate in Audulus. A pulse is a very short on/off burst, usually generated by clock modules, whereas gate signals can be any length from very short to very long.

In Audulus, no distinction is made between a pulse and a gate. That said, several clock modules give you control over the pulse width of their gate output. Pulse width is the ratio of on to off time. A 10\% pulse-width gate is on 10\% of the time and off 90\% of the time whereas a 90\% pulse-width gate is on 90\% of the time and off 10\% of the time.

\subsection{Modulation}

The Modulation signal is used to tweak parameters like filter cutoff, VCO shape, and VCA level. Its signal range is 0 to 1. 

Modulation signals are genereated by modules like LFOs, envelopes, sequencers, and sample \& holds. They can be used to modulate any knob on any module and often have their own separate inputs as well.

Modulation inputs and outputs are often marked with a red light. The light indicates the relative strength of the incoming modulation. If the light is black (not lit) its value is 0. If the light is fully red its value is 1.

When a module has several outputs like the Basic LFO, lights may be omitted to save CPU time. If a portion of a module does not terminate in a Meter node or audio output, then everything that precedes it will not be calculated. This means if you use only the Sine output of the Basic LFO, only the sine path will be calculated. If lights were present at each output they would force Audulus to calculate the unused paths, wasting CPU time.

Modulation inputs and outputs may be marked Mod, Env (Envelope), a specific waveshape like Sine or Saw, or correspond to a knob label like Hz Mod or Q. Sometimes they may be unmarked and have contextual meanings for each specific module, which are explained in their individual entries in the manual below.

\subsection{1/Octave}

The 1 per octave signal is a linearized pitch signal centered at 0 = A4 = 440Hz. To go an octave up, just add one: 1 = A5 = 880Hz. To go an octave down, subtract one: -1 = A3 = 220Hz. To go up or down in semitones, add or subtract in steps of 1/12th: 1/12 = A\sharp4 \ $\approx$ 466Hz; -1/12 = A\flat4 \ $\approx$ 415Hz. 

Hardware modular synthesizers typically use 1 volt per octave tracking where 0 volts is the lowest note, often C1. Audulus's 1/Octave signal is instead centered at the reference pitch, which is defaulted to A = 440Hz. 

The Keyboard node only outputs Hz, so make sure you use the MIDI Input module if you wish to use Audulus modules with a keyboard or pipe in a MIDI sequence from a DAW.

All non-gate sequencers in the Audulus module library generate a modulation signal only. This makes it easy to use these sequencers to modulate parameters as well as pitch. Their modulation output can be translated into a 1/Octave signal using a Modulation to 1/Oct utility module or with a quantizer module which all have this translation module built-in.

Translating a sequencer's output is simple. The Range parameter multiplies the 0 to 1 modulation output of the sequencer by 0 to 8. If the range is set to 2 then each knob of the sequencer covers 2 octaves. If the range is set to 0.5 then each knob covers 1/2 of an octave. The Shift parameter sets you lowest note. At 0, your lowest note will be A4. At -1, your lowest note will be A3. At 1, your lowest note will be A5.

The default reference pitch is set to A = 440Hz, but you can change this in any VCO if you wish. Enter any VCO and look for the 1/Oct to Hz converter (it may be a layer or two down in the module). Look for the RefHz variable and alter it to whatever pitch you wish. This change will only affect that particular VCO, so if you want to globally change the reference pitch you will have to make sure you do it in every VCO you use.

\subsection{Audio}

The Audio signal carries what you hear to your headphones or speakers. Its signal range is -1 to 1.

Audio signals are generated by VCOs (voltage-controlled oscillators) or an external audio input like a guitar or an audio track. They are modified by modules like VCFs (voltage-controlled filters), VCAs (voltage-controlled amplifiers), and effects like delay, reverb, and distortion.

Audio signals can be used as modulation at FM (frequency modulation) inputs of VCOs or as AM (amplitude modulation) at modulation inputs of VCAs. You can also modulate knobs at audio rates, but the knob will clip the negative portion of the audio signal. This means when the audio signal goes below 0, the knob will stay at 0. 

It is not recommended to plug an audio signal into a modulation input. Doing so may cause unpredictable and undesireable behavior from some modules. The only exception to this rule is the VCA modulation input, as mentioned above.

Audio inputs and outputs are unmarked by any lights. They may be labeled Audio, Aud, In/Out, Left/Right or L/R for stereo modules, or have context-specific labels like Sine or Saw on a VCO. Inputs labeled FM always expect an audio signal.

When audio signals are mixed together, their total output may exceed a -1 to 1 range. You can even boost your audio signals creatively to drive a distortion module harder. The important thing to remember is that audio exiting Audulus to your speakers, headphones, or audio interface must be kept between -1 and 1 to prevent clipping distortion. 

\section{Categories of Modules}

Broadly defined, there are 6 different categories of modules. If you understand what category a module is, you can start to understand how they are typically wired together. Of course once you understand conventional modular signal flow, you can subvert it and do things like stick a clock into a reverb and then into a modulation input of a VCA, but you'll be doing it because you're experimenting - not because you're clueless.

There are sometimes two or more versions of a particular module. The module that has a u- or $\mu$-prefix such as $\mu$LFO, $\mu$Clock, and $\mu$VCO are ``micro" modules. These are small, CPU-efficient versions of larger modules that have just bare essential functions. If you are completely new to modular synthesis, focus on using these modules first so you are not overwhelmed by the number of inputs, buttons, and controls of the other modules.

\subsection{Tempo}

Tempo modules set the speed of your patch. There is really only one type of tempo module: the clock module.

A clock provides a steady pulse that can tick forward a sequencer and be used to open and close an envelope. The Master Clock module has many different subdivisions to play around with, but if you're new to modular synthesis, you might want to first stick with the $\mu$Clock module.

\subsection{Beat}
\subsection{Pitch}
\subsection{Modulation}
\subsection{Audio-Generating}
\subsection{Audio-Modifying}


\section{Wiring Modules Together}
\subsection{Sequencer-Driven Subtractive Synthesizer}
\subsection{Keyboard-Controlled FM Synthesizer}
\subsection{Drum Patterns and Mixing}
\subsection{Stereo Effects}
\subsection{Generative Sequencing}
\subsection{Automation}
\subsection{DAW Integration}
\subsection{Eurorack Integration}


\chapter{Clock Modules}
\section{Clock Divider}
\section{Master Clock}
\section{Bernoulli Gate}
\section{Chance Select Gate}
\section{Pachinko Machine}
\section{Random Bursts}
\section{uClock}

\chapter{Drum Modules}
\section{uKick}

\chapter{Effect Modules}
\section{Delay}
\subsection{Stereo Delay}
\subsection{Looper Sync}
\subsection{Tempo Sync}
\subsection{uDelay}
\section{Distortion}
\subsection{Asymmetrical Drive}
\subsection{Fold Processor}
\section{Dynamics}
\section{Reverb}
\subsection{Really Humungous Reverb}
\subsection{uVerb}

\chapter{Envelope Modules}
\section{Analog Envelope}
\section{EOC Max ADSR}
\section{uADSR}

\chapter{Input-Output Modules}
\section{Audio Input}
\section{Audio Output}
\section{Expert Sleepers ES-3}
\section{Expert Sleepers ES-6}
\section{Expert Sleepers ES-8}
\section{MIDI Input}
\section{VPO Converter}

\chapter{LFO Modules}
\section{Basic LFO}
\section{Octature Sine LFO}
\section{TZFM LFO}
\section{uLFO}

\chapter{Mixer Modules}
\section{uMix Audio}
\section{uMix CV}

\chapter{Quantizer Modules}
\section{Chromatic Quantizer}
\section{Gateable Quantizer}
\section{Modulation Quantizer}
\section{Neo-Reimannian Triad Transformer}
\section{Quick Quantizer}

\chapter{Sequencer Modules}
\section{8 Step Sequencer}
\section{16 Step Gate Sequencer}
\section{Binary Gate Sequencer}
\section{Euclidean Gate Sequencer}
\section{Matrix Gate Sequencer}
\section{Shape Gate Sequencer}

\chapter{Utility Modules}
\section{Attenuate-Offset}
\section{Attenuverter}
\section{Audio Attenuverter-Offset}
\section{Automation Lane}
\section{Chaos Generator}
\section{Modulation to 1/Oct}
\section{Sample \& Hold}
\section{Slew}
\section{Switch Sequential 8 Step}
\section{VC Switch}
\section{uLogic}
\section{uSample \& Hold}
\section{Window Comparator}

\chapter{VCA Modules}
\section{Digital VCA}
\section{Diode VCA}
\section{uVCA}

\chapter{VCF Modules}
\section{Diode Ladder VCF}

\chapter{VCO Modules}
\section{Basic VCO}
\section{Chebyshev Additive VCO}
\section{Karplus-Strong VCO}
\section{Morphing VCO}
\section{Noise}
\section{Skew VCO}
\section{Supersaw VCO}
\section{TZFM VCO}
\section{uNoise}
\section{uVCO}







\end{document}
